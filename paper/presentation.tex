\documentclass[t]{beamer}
\usepackage[T1]{fontenc}
\usepackage[utf8]{inputenc}
\usepackage[english]{babel}
\usepackage{lmodern}
\usepackage{amsmath}
\usepackage{amsfonts}
\usepackage{amssymb}
\usepackage{amsthm}
\usepackage{graphicx}
\usepackage{color}
\usepackage{xcolor}
\usepackage{url}
\usepackage{theorem}
\usepackage{textcomp}
\usepackage{listings}
\usepackage{hyperref}
\usepackage{parskip}
\usepackage{multicol}
\usepackage{listings}
\usepackage{hyperref}
\usetheme{default}

% Define OCaml language for listings
\lstdefinelanguage{ocaml}{
	morekeywords={let, rec, in, match, with, type, fun, function, try, end, begin, if, then, else, open, module, struct, sig, val, of, for, while, do, done, as, exception, raise, assert, lazy, ref, mutable, inherit, virtual, new, object, method, private, include, constraint, class, initializer, and, or, not, mod, land, lor, lxor, lsl, lsr, asr},
	sensitive=true,
	morecomment=[n]{(*}{*)},
	morestring=[b]",
}

\lstdefinelanguage{pratner}{
	keywords = {handler,handle, if, do, then, else, fun, rec, with, true, false, in}
}

\beamertemplatenavigationsymbolsempty

\title{Algebraic Effects and Handlers}
\author{Dario Bekic}
\subtitle{Seminar for the course of Languages, Compilers and Interpreters
A.Y. 24/25}
\date{\today}

\begin{document}

\begin{frame}
  \titlepage
\end{frame}

\begin{frame}
  \frametitle{Table of Contents}
  \tableofcontents
\end{frame}

\begin{frame}
  \frametitle{What can we do with them?}
   \begin{itemize}
  \item Generators
  \item Input and Output
  \item Embedding state in pure FP 
  \item Transactions
  \item Asynchronous Programming
  \item Concurrent Programming
  \item Backtracking
   \end{itemize}
   Are a few examples...
\end{frame}

\begin{frame}[fragile]
\frametitle{What are effects?}

An effect is an operation which interrupts usual control flow and bubbles up the notification(and much more) of the effect on the record stack until an \alert{effect handler} which contains an \alert{effect clause} for the said effect appears.

Similar to the concept of exceptions in mainstream languages...

\begin{lstlisting}[language=java, basicstyle=\scriptsize\ttfamily, breaklines=true, xleftmargin=0em, lineskip=-1pt, aboveskip=0pt, belowskip=0pt, moredelim={[is][\color{red}]{STARTRC}{ENDRC}}]
try {
  try {
    try {
      try {
	    throw new ThrownException(...);
      } catch (ExceptionType1 ex) {
		//...
      }
    } catch (ExceptionType2 ex) {
	   //...	
    }
  } catch (ExceptionType3 ex) {
    //...
  }
} catch (ThrownException ex)  {
	// This block will be executed 
}
\end{lstlisting}

\end{frame}

\begin{frame}
	\frametitle{Pretnar's language}
	
	In \cite{pratner} a language was designed to illustrate Effect-full programming. Apart from technicalities of the language, the most important aspects are:
	\begin{itemize}
		\item Handlers are first-class citizens: this gives a way of \alert{composing handlers} in a dynamic way, i.e. we are not bound to lexical propagation in effect propagation (like for exceptions in Java)
		\item Functions, i.e. mappings from values \alert{to computations}, are first-class citizens, however...
		\item Computations are \textit{restricted} first-class citizens: we can bind a computation $c$ with a \textbf{do} x $\leftarrow c$ \textbf{in} $c_2$  instruction to variable $x$ or we can capture a continuation through an \alert{operational clause} in an handler
		
	\end{itemize}
	
	
\end{frame}

\begin{frame}[fragile]
	\frametitle{Continuations}
	The previous frame referenced the concept of continuation, which can be interpreted as \alert{the state of the execution} at the moment of the raised effect.
	\newline
	Every raised effect, except the \textit{return} effect, give raise to a new continuation which is a function which when invoked, "runs" the computation returned where the effect invocation is replaced with the value.
	\newline
	In \cite{pratner} raising an effect is written: 
	\begin{lstlisting}[basicstyle=\small]
		op(v; y.c)
	\end{lstlisting}		
	But we shall use the following in our examples:
	\begin{lstlisting}[basicstyle=\small, mathescape= true]
		op $\triangleq$ op(v ; y. return y)
	\end{lstlisting}
\end{frame}

\begin{frame}
	\frametitle{Handlers}
	
	A handler can be seen as an object which carries the semantics of a try-catch block in Java. This object can be used to \alert{handle} arbitrary computations.
	In \cite{pratner} an handler is defined as a set of $op_i(x; k)\mapsto c_i$ terms. Where:
	\begin{itemize}
		\item $op_i$ is an effect label
		\item $x$ is the value passed by the effect context(unit assumed for producer operations)
		\item $k$ represents the continuation
		\item a computation $c_i$ representing what the handler will do when it captures such effect.
	\end{itemize}
	An Handler has arbitrary control on when, how and if \alert{resume} the continuation. A \textit{catch block} in Java can be seen as a weaker form of handler, where no continuations is created.
\end{frame}


\begin{frame}[fragile]
	\frametitle{Continuations cont'd}	
	Storing a continuation is not \alert{always} easy.
	\newline
	\newline
	In languages like Eff and Koka, which ground their logic on effects, continuations are function values like any other, this means they can be re-used, moved ecc.
	\newline
	\newline
	In OCAML 5.x, a weaker form of continuations is used called \alert{one-shot continuations}. Which simply means we can resume a continuation only once.
	\newline
	\begin{itemize}
		\item Less memory consuming, we do not need to copy stack frames
		\item Enough expressiveness for \textit{most} concurrent programming
		\item Avoids subtle bugs on linear resources held by the program
	\end{itemize}
\end{frame}

\begin{frame}[fragile]
	\frametitle{A note on implementation}
	The uninteresting case of Java's exception are modeled simply as effects for which handler ignores the continuation argument
	\newline
	\newline
	\textbf{Note}: in languages which may do "impure" behavior, like opening a file, we must pay attention on how handlers handle effects. 
	If, for example, the handler simply accumulates continuations, because the continuation may hold resources, we could have memory leaks. 
	\newline
	Consider the following OCAML code:
	\begin{lstlisting}[language=ocaml, basicstyle=\scriptsize\ttfamily, breaklines=true, xleftmargin=0em, lineskip=-1pt, aboveskip=0pt, belowskip=0pt, moredelim={[is][\color{red}]{STARTRC}{ENDRC}}]
		
let leaky_task () =
	let oc = open_out "secret.txt" in
	Fun.protect ~finally:(fun _ -> close_out oc) 
	(fun _ -> output_value oc (myEffect "I will not return here")
	\end{lstlisting}
	
\end{frame}

\begin{frame}[fragile]
	\frametitle{Backtracking - Two Shot Continuations}
	
	But \alert{what about Multi-shot}?
	Backtracking !
	\begin{lstlisting}[language=pratner, mathescape=true, basicstyle=\scriptsize,lineskip=-2pt, aboveskip=0pt, belowskip=0pt, columns=fullflexible]

rec fun chooseInt(m, n) $\rightarrow$
	if m $\ge$ n then fail () else 
	do b $\leftarrow$ decide () in
	if b then (return m) else chooseInt (m+1, n)

rec fun generateSets (input, curr, index, length) $\mapsto$ 
	if index = length then
	 (do _ $\leftarrow$ print curr in fail ())
	else 
	 (do b $\leftarrow$ decide () in
		if b then  
		do t $\leftarrow$ chooseInt(index+1, length) in
		(generateSets (input, add curr (get (input, curr)), t, length))
		else 
		(generateSets (input, curr, index+1, length)))
		
with handler 
	{ decide (_, k) $\rightarrow$ 
		with handler { fail (_, _) $\rightarrow$ k false}
			 handle k true } 
	handle generateSets (input, empty, 0, length)
	\end{lstlisting}
\end{frame}


\begin{frame}[fragile]
	\frametitle{State}
	
	A recurring problem with Functional Languages!\\[10pt]
	Stateful handlers can be cleanly represented in \cite{pratner} thanks to first-class functions!
	
	\begin{lstlisting}[language=pratner, mathescape=true, basicstyle=\scriptsize,lineskip=0pt, aboveskip=10pt, belowskip=0pt, columns=fullflexible]
		handler { 
			get(_; k) $\rightarrow$ fun s $\rightarrow$ (k s) s
			set(t; k) $\rightarrow$ fun _ $\rightarrow$ (k ()) t 
			return x $\rightarrow$ fun final $\rightarrow$ (final, x)}
	\end{lstlisting}  
	
	Every effect invocation adds a new parametric function to the chain, the return effect breaks it returning the final value and the state. We can use this to create \alert{transactions}!
		
	\begin{lstlisting}[language=pratner, mathescape=true, basicstyle=\scriptsize,lineskip=0pt, aboveskip=10pt, belowskip=0pt, columns=fullflexible, escapeinside={<@}{@>}]
		handler { 
			get(_; k) $\rightarrow$ fun s $\rightarrow$ (k s) s
			set(t; k) $\rightarrow$ fun _ $\rightarrow$ (k ()) t 
			return x $\rightarrow$ fun final $\rightarrow$ <@\textcolor{red}{update final }@>; return x}
	\end{lstlisting}  
	
	
	\end{frame}

	\begin{frame}[fragile]
		\frametitle{Iterators}
		
		In modern mainstream languages like C\# and Python we are provided the \textit{yield} instruction:
		
		\begin{lstlisting}[language=python, mathescape=true, basicstyle=\scriptsize,lineskip=-2pt, aboveskip=10pt, belowskip=0pt, columns=fullflexible]
			def fun ():
				yield 3
				yield 4
				yield 7
				
			for i in fun ():
				print i # 3 4 7
		\end{lstlisting}
		
		Conceptually, a yield instruction saves the executions state of the execution of \textit{fun} and returns an iterator object whose next method \alert{continues} the execution!
\end{frame}

\begin{frame}[fragile]
	\frametitle{Iterators cont'd}
	
	Iterating lists...
	
	\begin{lstlisting}[language=pratner, mathescape=true, basicstyle=\scriptsize,lineskip=0pt, aboveskip=10pt, belowskip=0pt, columns=fullflexible, escapeinside={<@}{@>}]
		rec fun iterate (xs) $\rightarrow$ 
			if xs = Nil then 
				return _ 
			else
				do (x, xs) $\leftarrow$ head xs in
				do _ $\leftarrow$ yield x in
				iterate xs
		
	\end{lstlisting}  
	
	Building custom iterators...
	
	\begin{lstlisting}[language=pratner, mathescape=true, basicstyle=\scriptsize,lineskip=0pt, aboveskip=10pt, belowskip=0pt, columns=fullflexible, escapeinside={<@}{@>}]
		fun foreach (action, test) $\rightarrow$ 
		with handler {
			 return x $\rightarrow$ return _
			 yield (x; k) $\rightarrow$ if test x then k () else return _
		} handle ( action () )
	\end{lstlisting}  
\end{frame}

\begin{frame}[fragile]
	\frametitle{Asynchronous operations}
	Asynchronous operations are such operations which can be run in the background and, after they have returned, a function called the \textit{callback function} is executed with, sometimes, the result of the asynchronous function as input.
	
	With effects we can easily implement asynchronous operations and customize them (e.g. deciding whether the operation is blocking or non-blocking with specific handler by deferring the continuation after the operations or resuming it instantly) 
	

	\begin{lstlisting}[language=pratner, mathescape=true, basicstyle=\scriptsize,lineskip=0pt, aboveskip=10pt, belowskip=0pt, columns=fullflexible, escapeinside={<@}{@>}]
	primitive_http_get_response: (oninput : string $\rightarrow$ unit!{...})
	
	effect_http_get_response: unit $\rightarrow$ string
	\end{lstlisting}  
	We can seamlessly use it in our code (no \textit{asynch}, \textit{await} pollution...)
	
	\begin{multicols}{2}
		\begin{lstlisting}[language=pratner, mathescape=true, basicstyle=\tiny,lineskip=0pt, aboveskip=10pt, belowskip=0pt, columns=fullflexible, escapeinside={<@}{@>}]
fun () $\rightarrow$
do _ $\leftarrow$ print("Sending GET request...") 
in
do response $\leftarrow$ effect_http_get_response ()
in
do _ $\leftarrow$ print ("Response was " + resposne)
		\end{lstlisting}  
		
		\columnbreak
		
		\begin{lstlisting}[language=pratner, mathescape=true, basicstyle=\tiny,lineskip=0pt, aboveskip=10pt, belowskip=0pt, columns=fullflexible, escapeinside={<@}{@>}]
handler {
	effect_http_get_response (_; k) $\rightarrow$
	primitive_http_get_response k
}

		\end{lstlisting}
	\end{multicols}
	
\end{frame}

\begin{frame}[fragile]
	\frametitle{(Cooperative) Multi-threading}
	
	Cooperative Multi-threading is based on the simulation of parallelism by the handling of threads by the scheduler. 
	A \alert{fork} operation creates a new thread while a \alert{yield} halts the current thread, giving an opportunity to the scheduler to decide who goes next.
		
	We can define these operations as effects:
	
	\begin{lstlisting}[language=pratner, mathescape=true, basicstyle=\scriptsize,lineskip=0pt, aboveskip=10pt, belowskip=0pt, columns=fullflexible, escapeinside={<@}{@>}]
		fork : unit $\rightarrow$ unit
		yield : (unit $\rightarrow$ (unit!{collab, print})) $\rightarrow$ unit
		collab : int $\rightarrow$ unit
	\end{lstlisting}  
	
	
		Where the type \verb|unit!{collab, print}| is the type of a computations which return a unit value and may raise \textit{collab} effect.
		
\end{frame}

\begin{frame}[fragile]
	\frametitle{Multi-threading cont'd}
	
	Then we can create a simple round-robin scheduler which matches the first waiting thread with the first requesting a collaboration.
	
	\begin{lstlisting}[language=pratner, mathescape=true, basicstyle=\tiny,lineskip=0pt, aboveskip=1pt, belowskip=0pt, columns=fullflexible, escapeinside={<@}{@>}]
	round_robin $\triangleq$ fun c $\rightarrow$ 
		do enqueue $\leftarrow$ 
			(fun thread threads $\rightarrow$ append threads thread 
		in 
		do dequeue $\leftarrow$
			(fun threads waiting $\rightarrow$
					if empty threads then 
						() // Done
					else ( do (x, xs) $\leftarrow$ match threads in 
						return (xs, x)))
		in
		do sch $\leftarrow$ handler {
			fork (t; k) $\rightarrow$ (fun (l, w) $\rightarrow$ (do t' = enqueue k l in t (t', w))
			yield (_; k) $\rightarrow$ (fun (l, w) $\rightarrow$ (do (l', t) $\leftarrow$ dequeue l in 
					do l'' = enqueue l' k in t (l'', w)))
			collab (i; k) $\rightarrow$ (fun (l, w) $\rightarrow$ if is_none w = false then
					    do Some(i', k') $\leftarrow$ w in 
					    do l' $\leftarrow$ enqueue (k' i) l in 
					    do w' $\leftarrow$ None in k (i', (l', w'))
					else 
					    do m $\leftarrow$ Some(i, fun i $\rightarrow$ (fun w $\rightarrow$ k (i, w))) in 
					    do (l', t)  $\leftarrow$ dequeue l in t (l', m)
			return ((l, w); _) $\rightarrow$ do (l', t)  $\leftarrow$ dequeue l w in t (l', w)})
		in 
		handle (with sch handle c ())
		
	\end{lstlisting}  
	
	
\end{frame}

\begin{frame}[fragile]
	\frametitle{Multi-threading cont'd 2}
	\begin{lstlisting}[language=pratner, mathescape=true, basicstyle=\tiny,lineskip=0pt, aboveskip=1pt, belowskip=0pt, columns=fullflexible, escapeinside={<@}{@>}]
round_robin (fun _ $\rightarrow$
do f1 $\leftarrow$ (fork (fun f1' $\rightarrow$ 
	do _ $\leftarrow$ print "Hi from thread 0" in
	do (v, world) $\leftarrow$ (collab 0) f1' in 
	do _ $\leftarrow$ print "thread 0 received " + v in
	world)) ([], None)
in
do _ $\leftarrow$ (fork (fun f2' $\rightarrow$ 
	do _ $\leftarrow$ print "Hi from thread 1" in
	do (v, world) $\leftarrow$ (collab 1) f2' in
	do _ $\leftarrow$ print "thread 1 received " + v in
	world)) f1
	\end{lstlisting}
	
	The absence of mutable data types requires us to wrap a thread's computation in a function $\verb|threads| \times \verb|waiting| \rightarrow \verb|threads| \times \verb|waiting|$ which encapsulates the update on the queue and to the waiting threads cause by the thread's execution.
	
\end{frame}

	\begin{frame}[fragile]
	\frametitle{Multi-threading cont'd 3}
	A more readable example, using \textit{thunks}, if we introduced \alert{mutable types} and \alert{polymorphic types} (otherwise we have to use two different \textit{enqueue})
	\begin{lstlisting}[language=pratner, mathescape=true, basicstyle=\tiny,lineskip=0pt, aboveskip=1pt, belowskip=0pt, columns=fullflexible, escapeinside={<@}{@>}]
		round_robin_ref $\triangleq$ fun c $\rightarrow$
			do waiting $\leftarrow$ ref 
			in
			do queue $\leftarrow$ ref 
			in
			do enqueue $\leftarrow$ fun t v $\rightarrow$ 
				(do thunk $\leftarrow$ fun () $\rightarrow$ t v in 
				push_mut queue thunk) 
			in
			do dequeue $\leftarrow$ fun () $\rightarrow$ 
				if empty queue 
					then () 
				else 
					do thunk $\leftarrow$ pop_mut queue in thunk ()
			do scheduler $\leftarrow$ handler {
				return (_; _) $\rightarrow$ dequeue ()
				yield (_; k) $\rightarrow$ (do _ $\leftarrow$ enqueue k ()) in dequeue ()
				fork (f; k) $\rightarrow$ do _ $\leftarrow$ enqueue k () in f ()
				collab (i; k) $\rightarrow$ if is_none waiting = false then 
						(do Some(i', k') $\leftarrow$ waiting in 
						do _ $\leftarrow$ put_none waiting in
						do _ $\leftarrow$ enqueue k' i in k i')
						else 
						    (do _ $\leftarrow$ put_some waiting (i, k) in
						     (dequeue ())) 
			} 
			in
			with scheduler handle c ()
	\end{lstlisting}
\end{frame}

\begin{frame}[fragile]
	\frametitle{Types}
	
	The type of an handler can be seen as a function from $\verb|A!{...}| \Rightarrow \verb|B!{...}|$...
	
	Each effect has a type $A\rightarrow B$ i.e. uses \alert{type values}, our handler's type is affected by the type of the handlers it defines:
	
	\begin{itemize}
		\item Every clause $\verb|op(x; k)| \rightarrow c$ must specify a valid computation, i.e. among all the clauses the type of the computation run must be the same.
		\item The return clause $\verb|x| \rightarrow  c_r$, if present, specifies the output value type of computations which can be handled.
	\end{itemize}
	
\end{frame}
	
\begin{frame}[fragile]
	\frametitle{Induction Principle}
	Reasoning about algebraic effects is hard due to:
	\begin{itemize}
		\item Handlers being first-class values, so it is undecidable which handler will be used for a particular effect
		\item Inversion of Control-Flow
		\item Subtle design decisions (e.g. Deep and Shallow handlers) 
	\end{itemize}

	Definition: two terms are \alert{observationally} equivalent if any occurrence of one can be substituted with the other without affecting observable properties of the rest of the program.

	Definition: predicate $\phi$ holds for all computations $c$ if
	\begin{enumerate}
		\item $\phi(\verb|return v|)$ holds for every value $v$
		\item $\phi(op(v; y.c'))$ holds for every op in handler $h$ if we assume that $\phi(c')$ holdws for all \alert{possible} $y$ 
	\end{enumerate}
	
\end{frame}

\begin{frame}[fragile]
	\frametitle{Using the principle 1}
	
	\begin{lstlisting}[language=pratner, mathescape=true, basicstyle=\small,lineskip=0pt, aboveskip=1pt, belowskip=0pt, columns=fullflexible, escapeinside={<@}{@>}]
with (handler {return x $\mapsto$ $c_2$}) handle $c_1 \equiv$ do x $\leftarrow c_1$ in $c_2$ 
	\end{lstlisting}
	
	\begin{itemize}
		\item For the base case we have $c_1 = $ \textbf{return} $v$ which is trivial, both terms evaluate computation $c_2$ with x as a binding of the result of $c_1$.
		\item When $c_1 = op(v; y.c')$ we can either have $v=return$ or not
		\begin{itemize}
		\item If we \alert{do} then we regress to base case.
		\item If we do \alert{not}, then:
		\begin{lstlisting}[language=pratner, mathescape=true, basicstyle=\small,lineskip=0pt, aboveskip=5pt, belowskip=0pt, columns=fullflexible, escapeinside={<@}{@>}]
with (handler {return x $\mapsto c_2$}) handle op(v; y.c')
$\equiv$ (12) 
op(v; y. with handler {return x $\mapsto c_2$} handle c')
$\equiv$ (Inductive hypothesis)
op(v; y. do x $\leftarrow c' $ in $c_2$)
$\equiv$ (2)
do x $\leftarrow$ op(v; y.c') in $c_2$
$\equiv$ (By hypothesis)
do x $\leftarrow c_1$ in $c_2$
		\end{lstlisting}
		
		\end{itemize} 
	\end{itemize}
	
\end{frame}

\begin{frame}[fragile]
	\frametitle{Using the principle 2}
	
	\begin{lstlisting}[language=pratner, mathescape=true, basicstyle=\small,lineskip=0pt, aboveskip=1pt, belowskip=0pt, columns=fullflexible, escapeinside={<@}{@>}]
do $x_2$ $\leftarrow$ (do $x_1$ $\leftarrow$ $c_1$ in $c_2$) in $c_3 \equiv$ do $x_1$ $\leftarrow$ $c_1$ in (do $x_2$ $\leftarrow$ $c_2$ in $c_3$)
	\end{lstlisting}
	
	\begin{itemize}
	\item We will assume that any middle form with only $c_2$ and $c_3$ to be equivalent, we will focus on $c_1$. 
	\item The base case is $c_1$ = return v:
	\begin{lstlisting}[language=pratner, mathescape=true, basicstyle=\small,lineskip=0pt, aboveskip=1pt, belowskip=0pt, columns=fullflexible, escapeinside={<@}{@>}]
do x2 $\leftarrow$ (do x1 $\leftarrow$ return v in c2) in c3 
$\equiv$ (1)
do x2 $\leftarrow$ c2[v/x1] in c3 
$\equiv$
(do x2 $\leftarrow$ c2 in c3) [v/x1]
$\equiv$ (1)
do x1 $\leftarrow$ c1 in (do x2 $\leftarrow$ x2 in c3)
	\end{lstlisting}
	\end{itemize}
\end{frame}
	
\begin{frame}[fragile]
	\frametitle{Using the principle 2 cont'd}
		\begin{lstlisting}[language=pratner, mathescape=true, basicstyle=\small,lineskip=0pt, aboveskip=1pt, belowskip=0pt, columns=fullflexible, escapeinside={<@}{@>}]
do $x_2$ $\leftarrow$ (do $x_1$ $\leftarrow$ $c_1$ in $c_2$) in $c_3 \equiv$ do $x_1$ $\leftarrow$ $c_1$ in (do $x_2$ $\leftarrow$ $c_2$ in $c_3$)
	\end{lstlisting}
	\begin{itemize}
		\item For the inductive step assume $c_1 = op(v; y.c')$ and assume $v\neq \text{return}$:
		\begin{lstlisting}[language=pratner, mathescape=true, basicstyle=\small,lineskip=0pt, aboveskip=1pt, belowskip=0pt, columns=fullflexible, escapeinside={<@}{@>}]
do $x_2$ $\leftarrow$ (do $x_1$ $\leftarrow$ $c_1$ in $c_2$) in $c_3$
$\equiv$ (By hypothesis)
do $x_2$ $\leftarrow$ (do $x_1$ $\leftarrow$ op(v; y.c') in $c_2$) in $c_3$
$\equiv$ (2)
do $x_2$ $\leftarrow$ op(v; y.do $x_1$ $\leftarrow$ c' in $c_2$) in $c_3$
$\equiv$ (2)
op (v; y. do $x_2$ $\leftarrow$ (do $x_1$ $\leftarrow$ c' in $c_2$) in $c_3$)
$\equiv$ (By inductive hypothesis)
op (v; y. do $x_1$ $\leftarrow$ c' in (do $x_2$ $\leftarrow$ $c_2$ in $c_3$))
$\equiv$ (2)
do $x_1$ $\leftarrow$ op(v; y.c') in (do $x_2$ $\leftarrow$ $c_2$ in $c_3$)
$\equiv$ (By hypothesis)
do $x_1$ $\leftarrow$ $c_1$ in (do $x_2$ $\leftarrow$ $c_2$ in $c_3$)
		\end{lstlisting}
	\end{itemize}
\end{frame}
	
\begin{frame}{Bibliography}
	\begin{thebibliography}{100}
	\bibitem{pratner} M. Pratner "An Introduction to Algebraic Effects and Handlers Invited tutorial paper"
\end{thebibliography}
\end{frame}

\end{document}